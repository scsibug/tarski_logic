\section{On the use of variables}
\begin{enumerate}
\item Which among the following expressions are sentential functions,
  and which are designatory functions:
  \begin{enumerate}
  \item \emph{x is divisible by 3}.

    Sentential, since $x$ is a free variable, and the divisibility of
    a number by another is an assertion.

  \item \emph{the sum of the numbers x and 2}.

    Designatory, since the sum of $x$ and 2 describes a number through
    addition.

  \item $y^2-z^2$

    Designatory, since the expression describes a number through
    exponentiation and subtraction.

  \item $y^2=z^2$

    Sentential, expressing the assertion that the square of $x$ and
    $z$ are equal.

  \item $x+2<y+3$

    Sentential, expressing an inequality.

  \item $(x+3)-(y+5)$

    Designatory, describes a value through addition and subtraction of
    free variables.

  \item \emph{the mother of x and z}.

    Designatory, describes a person.

  \item \emph{x is the mother of z} \hspace{1em}.

    Sentential, makes an assertion about $x$ and $z$.

  \end{enumerate}

\item Give examples of sentential and designatory functions from the
  field of geometry.
  \begin{enumerate}
  \item \textbf{Sentential:} The area of circle $a$ is twice that of
    square $b$.
  \item \textbf{Designatory:} The circle having a radiuz of $z$.
  \end{enumerate}


\item The sentential functions which are encountered in arithmetic and
  which contain only one variable (which may, however, occur at
  several different places in the given sentential function) can be
  divided into three categories: (i) functions satisfied by every
  number; (ii) functions not satisfied by any number; (iii) functions
  satisfied by some numbers, and not satisfied by others.

  To which of these categories do the following sentential functions
  belong:

  \begin{enumerate}
  \item $x+2=5+x$

    Category (ii).  No number is equal to itself plus 3.

  \item $x^2=49$

    Category (iii).  $x=7$ satisfies the formula, but $x=8$ does not.

  \item $(y+2)\times(y-2)<y^2$

    Category (i).  The left-hand side of the inequality expands to
    $(y^2-4)$.  This is obviously always less than the right hand
    side, and so the inequality will always hold true.

  \item $y+24>36$

    Category (iii).  $y>8$ satisfies the formula, while $y=0$ does
    not.

  \item $z=0$ \emph{or} $z<0$ \emph{or} $z>0$

    Category (i). All numbers meet one of the given criteria.

  \item $z+24>z+36$ \hspace{1em}

    Category (ii).  No number is greater than itself plus 8.

  \end{enumerate}

\item Give examples of universal, absolutely existential and
  conditionally existential theorems from the fields of arithmetic and
  geometry.
  \begin{enumerate}
  \item Universal

    \textbf{Arithmetic:} Associative law, $\forall a, b, c :
    (a+b)+c=a+(b+c)$.

    \textbf{Geometry:} For all circles, $C=\pi \times d$.

  \item Absolutely existential

    \textbf{Arithmetic:} $\exists a : a > 5$.

    \textbf{Geometry:} There exists a triangle with internal angles
    measuring $30^{\circ}$, $60^{\circ}$ and $90^{\circ}$.

  \item Conditionally existential

    \textbf{Arithmetic:} For all $x$ and $y$, there is a $z$ such that
    $z = x + y$. % from page 9

    \textbf{Geometry:} Given any straight line and a point not on it,
    there exists one and only one straight line which passes through
    that point and never intersects the first line, no matter how far
    they are extended. (Parallel Postulate)

  \end{enumerate}

\item By writing quantifiers containing the variables ``$x$'' and
  ``$y$'' in front of the sentential function: $$ x > y $$ it is
  possible to obtain various sentences from it, for instance:
\begin{equation*}
\emph{for any numbers x and y, } x > y;
\end{equation*}
\begin{equation*}
\emph{for any number x, there exists a number y such that } x > y;
\end{equation*}
\begin{equation*}
\emph{there is a number y such that, for any number x, } x > y.
\end{equation*}

Formulate them all (there are six altogether) and determine which of
them are true.

\begin{enumerate}
\item $\emph{for any numbers x and y, } x > y.$
  \begin{itemize}
  \item False, since $5 \ngtr 10$.
  \end{itemize}
\item $\emph{for any number x, there exists a number y such that } x > y.$
  \begin{itemize}
  \item True, take the value bound to $x$, and set $y = x - 1$.
  \end{itemize}
\item $\emph{for any number y, there exists a number x such that } x > y.$
  \begin{itemize}
  \item True, take the value bound to $y$, and set $x = y + 1$.
  \end{itemize}
\item $\emph{there is a number y such that, for any number x, } x > y.$
  \begin{itemize}
  \item False, since there is no smallest number.
  \end{itemize}
\item $\emph{there is a number x such that, for any number y, } x > y.$
  \begin{itemize}
  \item False, since there is no largest number.
  \end{itemize}
\item $\emph{there are numbers x and y, such that } x > y.$
  \begin{itemize}
  \item True, $4 > 5$.
  \end{itemize}
\end{enumerate}

\item Do the same as in Exercise 5 for the following sentential
  functions:
  \begin{equation*}
    x + y^2 > 1
  \end{equation*}
  and
  \begin{equation*}
    \emph{x is the father of y}
  \end{equation*}
  (assuming that the variables ``$x$'' and ``$y$'' in the latter stand
  for names of human beings).

  \begin{enumerate}
  \item $\emph{for any numbers x and y, } x + y^2 > 1$
    \begin{itemize}
    \item False, since $0 + 0^2 \ngtr 1$.
    \end{itemize}
  \item $\emph{for any number x, there exists a number y such that } x
    + y^2 > 1$
    \begin{itemize}
    \item True, take the value bound to $x$, and set $y = |x| + 2$.
    \end{itemize}
  \item $\emph{for any number y, there exists a number x such that } x
    + y^2 > 1$
    \begin{itemize}
    \item True, the formula is satisfied whenever $x \geq 2$.
    \end{itemize}
  \item $\emph{there is a number y such that, for any number x, } x +
    y^2 > 1$
    \begin{itemize}
    \item False, take the value bound to $y$, and set $x = - y^2$.
    \end{itemize}
  \item $\emph{there is a number x such that, for any number y, } x +
    y^2 > 1$
    \begin{itemize}
    \item True, $x = 2$.
    \end{itemize}
  \item $\emph{there are numbers x and y, such that } x + y^2 > 1$
    \begin{itemize}
    \item True, $1 + 2^2 > 1$.
    \end{itemize}
  \end{enumerate}
  \vspace{1em}
  \begin{enumerate}
  \item \emph{for any x and y, x is the father of y}
    \begin{itemize}
    \item False.
    \end{itemize}
  \item \emph{for any x, there exists a y such that x is the father of
      y}
    \begin{itemize}
    \item False, not every person is a father.
    \end{itemize}
  \item \emph{for any y, there exists an x such that x is the father
      of y}
    \begin{itemize}
    \item True, every person has a father.
    \end{itemize}
  \item \emph{there is a y such that, for any x, x is the father of y}
    \begin{itemize}
    \item False, not every person has the same father.
    \end{itemize}
  \item \emph{there is an x such that, for any y, x is the father of
      y}
    \begin{itemize}
    \item False, no one is the father of everyone else.
    \end{itemize}
  \item \emph{there exist x and y such that, x is the father of y}
    \begin{itemize}
    \item True, there exists some father-child relationship.
    \end{itemize}
  \end{enumerate}

\item State a sentence of everyday language that has the same meaning
  as:
  \begin{equation*}
    \emph{for every x, if x is a dog, then x has a good sense of smell}
  \end{equation*}
  and that contains no quantifier or variables.

  \begin{itemize}
  \item Dogs have a good sense of smell.
  \end{itemize}

\item Replace the sentence:
  \begin{equation*}
    \emph{some snakes are poisonous}
  \end{equation*}
  by one which has the same meaning but is formulated with the help of
  quantifiers and variables.

  \begin{itemize}
  \item There exists an $x$, such that $x$ is a snake, and $x$ is
    poisonous.
  \end{itemize}

\item Differentiate, in the following expressions, between the free
  and bound variables:
  \begin{enumerate}
  \item \emph{x is divisible by y.}
    \begin{itemize}
    \item \textbf{Bound:} none
    \item \textbf{Free:} $x$, $y$
    \end{itemize}
  \item $\emph{for any x,} \enspace x - y = x + (-y).$
    \begin{itemize}
    \item \textbf{Bound:} $x$
    \item \textbf{Free:} $y$
    \end{itemize}
  \item $\emph{if } \enspace x < y, \enspace \emph{then there is a
      number z such that } \enspace x < y \enspace \emph{and} \enspace
    y < z.$
    \begin{itemize}
    \item \textbf{Bound:} $z$
    \item \textbf{Free:} $x$, $y$
    \end{itemize}
  \item $\emph{for any number y, if} \enspace y > 0, \enspace
    \emph{then there is a number z such that} \enspace x = y \cdot z.$
    \begin{itemize}
    \item \textbf{Bound:} $y$, $z$
    \item \textbf{Free:} $x$
    \end{itemize}
  \item $\emph{if} \enspace x = y^2 \enspace \emph{and} \enspace y >
    0, \enspace \emph{then, for any number z,} \enspace x > -z^2.$
    \begin{itemize}
    \item \textbf{Bound:} $z$
    \item \textbf{Free:} $x$, $y$
    \end{itemize}
  \item $\emph{if there exists a number y such that} \enspace x > y^2,
    \enspace \emph{then, for any number z,} \enspace x > -z^2.$
    \begin{itemize}
    \item \textbf{Bound:} $y$, $z$
    \item \textbf{Free:} $x$
    \end{itemize}
  \end{enumerate}

  Formulate the above expressions by replacing the quantifiers by the
  symbols introduced in Section 4.

  \begin{enumerate}
  \item \emph{x is divisible by y.}
  \item $\forall x, \enspace x - y = x + (-y).$
  \item $x < y \rightarrow \exists z, \enspace x < y \enspace
    \emph{and} \enspace y < z.$
  \item $\forall y, \enspace y > 0 \rightarrow \exists z, \enspace x =
    y \cdot z.$
  \item $x = y^2 \emph{ and } y > 0 \rightarrow \forall z, \enspace x
    > -z^2.$
  \item $\exists y, \enspace x > y^2 \rightarrow \forall z, \enspace x
    > -z^2.$
  \end{enumerate}

\item If, in the sentential function (e) of the preceding exercise, we
  replace the variable ``$z$'' in both places by ``$y$'', we obtain an
  expression in which ``$y$'' occurs in some places as a free and in
  others as a bound variable; in what places and why?

  \begin{equation*}
    x = y^2 \emph{ and } y > 0 \rightarrow \forall y, \enspace y > -z^2.
  \end{equation*}

  The first and second occurrences of $y$ are free, but occurrences
  after the implication are bound by the universal quantifier.

\item Try to state quite generally under which conditions a variable
  occurs at a certain place of a given sentential function as a free
  or bound variable.

  A variable is bound whenever it appears after a quantifier has been
  applied to it.  If a variable appears, but no quantifier directly
  refers to it yet in the formula (to the left of the occurrence),
  then it is free.

\item Which numbers satisfy the sentential function:
  \begin{equation*}
    \emph{there is a number y such that} \enspace x = y^2
  \end{equation*}
  \begin{itemize}
  \item For $x$, the squares of all numbers in the universe of
    discourse.
  \end{itemize}
  and which satisfy
  \begin{equation*}
    \emph{there is a number y such that} \enspace x \cdot y = 1
  \end{equation*}
  \begin{itemize}
  \item For $x$ and the universe of discourse being $\mathbb{N}$, only
    1 satisfies the function.
  \end{itemize}

\end{enumerate}
